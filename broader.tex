\section{Broader Impacts}
\label{sec:broader}

The proposed research is expected to have important impacts on the built environment.
According to the EPA, buildings are responsible for producing 6\% of
greenhouse gasses and heat and electrical generation produces another
25\%\footnote{\url{https://www.epa.gov/ghgemissions/global-greenhouse-gas-emissions-data}}.
A large portion of the electrical consumption in buildings is used for
heating, cooling and artificial lighting. Our proposal addresses the
reduction of electricity for artificial lighting to be replaced by
reflected daylight and capturing the solar heat. During daytime hours,
daylight is a preferred source of light for many people and the proposal
directs the daylight deeper into a building. The daylight reflection system
provides a more sustainable approach by reducing the required electricity
and provides a more desirable quality of light.

Catoptric systems have the potential to make plant growth in controlled
environments more sustainable and efficient. Different plant species have
different light requirements and there is a balance between light limitation,
optimum light, and light stress growth conditions. Plant breeding programs
often target plant architecture that will optimize light capture through a
canopy~\cite{burgess17} and light penetration through the canopy has been
shown to be important for plant defense to pathogens~\cite{ballare12}.

In addition to increased sustainability of controlled plant growth in
professional
settings, we plan to include these catoptric systems in education and
outreach activities conducted across the St.~Louis region. Dr.~Kristine
Callis-Duehl is the Director of Education and Outreach at the Donald Danforth
Plant Science Center and is a senior member of this planning grant. Dr.
Callis-Duehl will help to incorporate this catoptric system in education
activities included in the resulting grant. To test the feasibility of using
the catoptric system in education, it will be used in an
existing weeklong summer camp program at the Danforth Center. Participants will
help to program mirror movement to optimize growth of their own plant, which
will allow for integrated lessons in engineering, computer science, and plant
science.  

Finally, we will collaborate with colleagues from the Washington University
Brown School of Social Work and Public Health to investigate ways in which
vegetables grown with the help of catoptric systems can be used to improve
health in urban food deserts~\cite{Walker10}.

