\section{Broader Impacts}
\label{sec:broader}

The proposed research is expected to have important impacts on the built environment.
According to the EPA, buildings are responsible for producing 6\% of
greenhouse gasses and heat and electrical generation produces another
25\%\footnote{\url{https://www.epa.gov/ghgemissions/global-greenhouse-gas-emissions-data}}.
A large portion of the electrical consumption in buildings is used for
heating, cooling and artificial lighting. Our proposal addresses the
reduction of electricity for artificial lighting to be supplemented or even
replaced by
reflected daylight and also capturing solar heat. During daytime hours,
daylight is a preferred source of light for many people, and by
directing daylight deeper into a building, the catoptric system
provides a more sustainable approach by reducing the required electricity
and providing a more desirable quality of light.

Different plant species have
different light requirements and there is a balance between light limitation,
optimum light, and light stress growth conditions. Plant breeding programs
often target plant architecture that will optimize light capture through a
canopy~\cite{burgess17} and light penetration through the canopy has been
shown to be important for plant defense to pathogens~\cite{ballare12}.
Catoptric systems also have the potential to make plant growth in controlled
environments more sustainable and efficient.
Further, there are often discrepancies between plant phenotypes observed
in controlled growth environments in comparison to field settings,
and light limitation and light quality in controlled environments likely
contributes to these differences. If catoptric systems can capture and
better mimic field settings it may be used to close the gap between field
and controlled environment research.

In addition to increased sustainability of controlled plant growth in
professional
settings, we plan to include these catoptric systems in education and
outreach activities conducted across the St.~Louis region. Dr.~Kristine
Callis-Duehl, Director of Education and Outreach at the Donald Danforth
Plant Science Center, 
will help to incorporate this catoptric system in education
activities of the Center. To test the feasibility of using
the catoptric system in education, it will be used in an
existing weeklong summer camp program at the Danforth Center. Participants will
help to program mirror movement to optimize growth of their own plant, which
will allow for integrated lessons in engineering, computer science, and plant
science.  

Finally, we will collaborate with Dr. Lora Iannotti and Dr. Melissa Jonson-Reid
of the Washington University
Brown School of Social Work and Public Health to investigate ways in which
the affordable local production of
vegetables grown with the help of catoptric systems can be used to improve
health in urban food deserts~\cite{Walker10}.

