\documentclass[12pt]{article}
\usepackage{epsfig}
\usepackage{subfigure}
\usepackage{setspace}
\usepackage{url}

\setlength{\textheight}{8.875in} \setlength{\textwidth}{6.5in}
\setlength{\topmargin}{0.0in} \setlength{\headheight}{0.0in}
\setlength{\headsep}{0.0in} \setlength{\oddsidemargin}{0.0in}

\begin{document}
\pagestyle{empty}
\thispagestyle{empty}

\begin{center}
\textbf{\Large Data Management Plan}
\end{center}

This project will produce design software (both for low-level positioning
control and higher-level optimization)
that we fully intend to release as an open-source resource that
is usable by the community at large.

Empirical testing and performance evaluation will be strongly data
driven.  We will make the data from the reliability testing of the
individual mirror units available as a public-domain data set.

Novel performance analysis data (e.g., the results of experiments
that speak to how one or more candidate designs will perform under
a given set of circumstances) will be published as part of the
project's publications.

Materials for distribution will be made available through the PIs' lab
website in the Department of Computer Science and Engineering at
Washington University in St.\ Louis (sbs.wustl.edu).
Departmental web servers
are professionally maintained by the School of Engineering and Applied
Science IT staff.
We will ensure that any material posted to this site will be available
through the end of the full project period and for at least three
years thereafter.

In addition, research materials will be made available through the
Open Scholarship Digital Research Materials Repository of the 
Washington University Libraries.  This repository provides persistent,
curated access to both code and data, including support for DOIs.

We will review all material provided by our group to ensure that it is
not encumbered by its authors' intellectual property protections in a
way that would prevent open-source distribution.

\end{document}
