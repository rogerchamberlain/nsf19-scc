\documentclass[11pt]{article}
\usepackage{epsfig}
\usepackage{subfigure}
\usepackage{setspace}
\usepackage{url}

\setlength{\textheight}{8.875in} \setlength{\textwidth}{6.5in}
\setlength{\topmargin}{0.0in} \setlength{\headheight}{0.0in}
\setlength{\headsep}{0.0in} \setlength{\oddsidemargin}{0.0in}

\begin{document}
\pagestyle{empty}
\thispagestyle{empty}

\begin{center}
\textbf{\Large Facilities, Equipment, and Other Resources}
\end{center}

\noindent
This project will be conducted across three organizations:
Washington University
in St.~Louis, the Dept. of Computer Science at Saint Louis University,
and the Donald Danforth Plant Science Center.
Beyond normal academic office space and computer equipment, the 
specific requirement needed to carry out the proposed research is access to
suitable physical fabrication and testing facilities.

\subsection*{Laboratory}

At Washington University in St. Louis, the project will be housed
in the Stream Based Supercomputing
(SBS) laboratory, which has laboratory
facilities on the 5th floor of Jolley Hall,
and in the Graduate School of Architecture, which has laboratory
facilities in Givens and Steinberg Halls.

PIs at the Danforth Center each have between 1500 and 4000 sq.~ft.~of laboratory
space and are provided office space within or adjacent to their laboratory
space.
The laboratories are completely outfitted with lab benches, fume hoods,
walk-in cold rooms, ultrapure water systems, space for large equipment and
office space for students and staff. General use equipment such as high speed
centrifuges, ultracentrifuges, spectrophotometers, fluorometers,
phosphorimagers, darkrooms, autoclaves, ice machines, and dishwashers are
available on each floor of the facility. Specialty equipment and materials such
as freeze driers, vacuum centrifuges, liquid nitrogen and dry ice, are
available in centrally located positions within the building. Each investigator
has ultralow temperature freezers, incubators, thermal cyclers, waterbaths,
etc. in their own laboratories.

\subsection*{Office Space}

The Washington University 
Dept.\ of Computer Science and Engineering and the Graduate School of
Architecture provide office space to
all faculty, staff, postdocs, and graduate students.  The PIs have office
space in Jolley Hall and in Givens Hall,
while the graduate students will be housed in student offices
within Jolley Hall and Steinberg Hall.
Equivalent office space is available to faculty and students at Saint
Louis University.

\subsection*{Computer Equipment}

All computer equipment associated with the WU Dept.\ of Computer Science
and Engineering is supported by Engineering Information Technology
(EIT), an organization of the McKelvey School of Engineering.
The EIT staff includes full-time system
administrators and hardware technicians.

The Danforth Center supports computing through several modalities: 1)
high-performance computing and workflow management on an HTCondor cluster; 2)
virtualized applications using machine- and container-level virtualization; 3)
web/database applications and support. Currently, the infrastructure contains
over 1,300 processors and 2,800 graphics processors, more than 8~TB of
memory, and a single, high-performance 540 terabyte storage area network. These
resources are shared in a managed, multi-user environment and communicate via a
10 gigabit ethernet network.

\subsection*{WU Laboratory Equipment and Software}

The SBS laboratory has bench power supplies, oscilloscopes, logic
analyzers, multimeters, and function generators.  Available CAD
software includes Mentor Graphics FPGAdvantage, Synplicity Synplify
Pro, Xilinx ISE and Vivado Design Suite, Altera Quartus, Cadence,
HSPICE, Virtuoso, and Alegro.  For GPU operations, we have the latest
versions of the NVIDIA Linux kernel drivers and development toolkits.

\vspace{0.1in}
\noindent
The Graduate School of Architecture has a full metalworking and woodworking
shop, including a number of 3D printers, an XYZ 3-axis CNC router,
and 4 CNC laser cutting machines.  Available software
includes Rhino3D and Grasshopper.

\subsection*{Specialized Equipment at the Danforth Center}

The Danforth Center Plant Growth Facility (PGF) is comprised of three
greenhouse ranges and two environmental growth chamber areas. In total, the PGF
includes 42 individual greenhouses and 84 Conviron growth chambers and growth
rooms. The total area for the greenhouses is approximately 53,500 square feet.
The greenhouses are of
various sizes ranging from 400 to 2,100 sq.~ft.~and offer several different
functional capabilities. All greenhouses are environmentally controlled for
temperature, humidity and light levels using Argus Control Systems. Specialized
growth areas include five air-conditioned greenhouses, high light greenhouses,
and propagation houses. Select houses are 14 ft. to the gutter for growing
taller crops with in-floor heating as well as several houses with
height-adjustable light canopies and auto-irrigation capabilities.

The PGF includes a combined 84 Conviron plant growth chambers and rooms,
providing 5,795 square feet of controlled environment plant growth area.
Forty-eight (48) reach-in growth chambers range in size from 14 to 57 sq. ft. of
growth area and are available in single, double, or triple tier conformations.
Thirty-six (36) walk-in growth rooms range in size from 36 to 150 sq. ft. of
growth area.  All units offer variable lighting, temperature and humidity
control. The majority of chambers are equipped with dimmable fluorescent
lighting, while 8 walk-in rooms utilize high intensity discharge (HID)
lighting, and 2 specialized reach-ins offer spectral control with Heliospectra
LED lighting systems. Additional specialty capabilities include low temperature
control to 4C (13 reach-ins, 3 walk-ins), sub-freezing temperature control to a
minimum of -10C (one 80 sq. ft. walk-in), and maximum temperature control to 45C
or 50C in all units. Two reach-ins offer additive CO$_2$ control to 3000 ppm,
while two 80 ft. walk-ins and four reach-ins possess both additive CO$_2$ (max 3000
ppm) and CO$_2$ scrubbing (down to ~100-125 ppm) capability. Many chambers and
rooms offer high light intensity control of at least 500 $\mu$mol/m$^2$/s up to a
maximum of 1,100 $\mu$mol/m$^2$/s.  Controllable electrical receptacles, ethernet
network connectivity, programmable irrigation valves, and multiple instrument
ports in a majority of units allow for accommodation of auxiliary equipment
(e.g. cameras, sensors, hydroponic systems, etc.).

This project utilizes two custom-built image-based phenotyping platforms in the
Gehan Lab at the Danforth Center: 1) the Hyperspectral Development Platform;
and 2) Two Microcomputer-Fitted Growth Chambers for high-throughput imaging.

The Hyperspectral Development Platform (HDP) custom designed for the Gehan lab
at the Danforth Center includes a high-resolution VNIR camera (Headwall Series
E, 380 to 1000 nm, with 1000 spectral bands) and three time-of-flight cameras
integrated with an industrial-scale robotic arm for automated camera
positioning. The HDP can image individual plants up to 2 meters tall (6.6 feet)
at 0.5 mm spatial resolution. The system has dedicated illumination from two
angles.

For the Microcomputer-Fitted Growth Chambers, two Conviron growth chambers with
150 sq. ft. of growth area each were outfitted with 144 Raspberry Pi
microcomputers and 8 megapixel RGB cameras. Both growth chambers have a light
intensity range from 0 to 400 $\mu$mol/m$^2$/s and one of the growth chambers has a
temperature range from 4C to 45C, while the other ranges from 10C to 45C.
Approximately 1,296 4-inch pots can be imaged in each chamber simultaneously
(2,592 total) at high-frequency (every minute if needed).

\end{document}
