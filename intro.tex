\section{Introduction}
\label{sec:intro}

%\FIXME{Here is a reference to make sure that citations are working~\cite{Bugbee16,bblmw19}.}

Energy consumption due to buildings (both residential and commercial)
is estimated to be 20\% to 40\% of the total energy usage in
developed countries~\cite{pop08}, and
lighting and heating are two significant components of this demand~\cite{keh05}.
Natural light (i.e., sunlight) is a readily available resource that
can contribute both to the illumination~\cite{Leslie03}
and to the heating~\cite{Lunde80} of structures.
Further, the use of natural light for plant growth is dramatically more
energy efficient than artificial lighting~\cite{Bugbee16}. \FIXME{Need reference.}
For example, supplemental lighting at the Donald Danforth Plant Science Center
greenhouses consumes ... [lots of energy, dollars, something] \FIXME{Malia, can you complete this sentence? Or replace it with something that says something similar?}

We propose to investigate how to utilize actively
controlled catoptric (mirror) surfaces effectively for improved illumination,
plant growth, and heating of buildings.  Through computer-based control of
the dynamic positioning of
individual mirrors, and cyber-physical integration of the mirrors,
the devices
that orient them, and customized objectives and
constraints,
we propose to enable fine-grained management of sunlight as a resource.

Figure~\ref{fig:amp} shows a prototype catoptric surface (called AMP) that was 
designed, fabricated, and installed during an undergraduate architecture studio 
taught by Co-PI C.~Ahrens. The installation redirects light from gable ends of an 
existing building into the darker recesses of the atrium.
We next developed a
new version in which over 600 mirrors are under 
active, 2-axis control and therefore the mirrors
can be pointed in different directions dynamically as desired over
time~\cite{acmbg19,acmb18,cagm18}. 
This next-generation installation is within 
the south wall of the Steinberg Hall atrium on the campus of 
Washington University; a subset of the mirrors in this new
installation is shown in Figure~\ref{fig:steinberg}.

\begin{figure}[ht]
\centering
\subfloat[\mbox{ }]{
\includegraphics[width=0.45\linewidth]{figures/amp}
\label{fig:amp}}
\qquad \qquad
\subfloat[\mbox{ }]{
\includegraphics[width=0.14\linewidth]{figures/steinberg}
\label{fig:steinberg}}
\caption{Catoptric system prototypes.
(a)~\emph{AMP}, TRex building, St.~Louis.
(b)~Steinberg Hall, St.~Louis.
}
\label{fig:proto}
\end{figure}

The community with which we will partner on this effort is
the 39~North AgTech Innovation District in St.~Louis County. Under the
supervision of the county and the St.~Louis Economic Development Partnership, 
a master plan has been prepared to develop the area in the vicinity of  
anchor institutions Bayer AG and the Donald Danforth Plant Science Center. 
This includes greenspaces, startup incubators, redesigned roadways, etc.,
all in support of new agriculture technology businesses and plant science.
