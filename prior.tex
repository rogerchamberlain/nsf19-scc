\section{Results from Prior NSF Support}
\label{sec:prior}

Co-PI C. Ahrens has not had prior NSF funding. The projects described
below are representative examples of NSF-supported research led by PI 
R.~Chamberlain and co-PI C.~Gill.

\noindent
{\large\bf CSR: Small: Concurrent Accelerated Data Integration}
{\bf (CNS-1527510,
PI R. Chamberlain)}, 
10/2015--9/2019, \$519,275.  

\textbf{Intellectual Merit} -- This project is investigating the
accelerated execution of data integration workflows, which
increasingly are bottlenecks in data science. Execution platforms
include both graphics engines and FPGAs.

\textbf{Broader Impacts} -- This project has supported 5
graduate students and 6 REU students.  The applications investigated
come from computational biology, astrophysics, and the IoT,
further expanding the scope of the students'
experience.

\textbf{Evidence of Research Products and their Availability} --
Publications resulting from this work include~\cite{cc19,dibs,c17,fcbmc19,mgc16,js16}.
A benchmark suite of the above workflows has been released
as a community resource~\cite{dibsv1}.

\noindent
{\large\bf CPS: Medium: Collaborative: CyberMech, a Novel Run-Time Substrate for 
Cyber-Mechanical Systems}
{\bf (CNS-1136073 and CNS-1136075,
WU PI C. Gill)}, 9/2011-8/2016, \$1,800,000 total.  

This research project developed novel foundations for parallel real-time computing, demonstrating the first ever real-time hybrid simulation involving a thousand-degree-of-freedom structure at millisecond time scales.

\textbf{Intellectual Merit} -- Results of this research include new methods for parallel real-time execution of control and simulation computations, new parallel real-time scheduling techniques and analyses, and characterization and exploitation of trade-offs involving both high computational demand and stringent timing constraints.

\textbf{Broader Impacts} -- This multi-university project involved 7 PhD, 3 masters, and 7 undergraduate students, and 2 visiting scholars in highly multi-disciplinary research collaborations.  Results of this research are spurring discussions on scalability of real-time hybrid simulation techniques, at workshops on topics of current interest in the earthquake engineering research community, including at a recent NSF-sponsored workshop at UCSD on multi-hazard engineering.

\textbf{Evidence of Research Products and their Availability} -- Results of this 
collaborative research appeared in 10 publications at top-tier conferences and journals.
Data, experiment configurations, platform software, and simulation source-code 
have been published on-line at Washington University and Purdue University.

