\section{Results from Prior NSF Support}
\label{sec:prior}

Faculty Co-PIs C. Ahrens and A. Stylianou have not had prior NSF funding.
The projects described
below are representative examples of NSF-supported research led by PI 
R.~Chamberlain and co-PI C.~Gill.

%\noindent
{\bf CSR: Small: Concurrent Accelerated Data Integration}
{\bf (CNS-1527510,
PI R. Chamberlain)}, 
10/2015--9/2019, \$519,275.  
%
\textbf{Intellectual Merit} -- This project is investigating the
accelerated execution of data integration workflows, which
increasingly are bottlenecks in data science. Execution platforms
include both graphics engines and FPGAs.
%
\textbf{Broader Impacts} -- This project has supported 5
graduate and 6 REU students.  The applications investigated
come from biology, astrophysics, and the IoT,
further expanding the scope of the students'
experience.
%
\textbf{Evidence of Research Products and their Availability} --
Publications resulting from this work include~\cite{cc19,dibs,c17,fcbmc19,mgc16,js16}.
A benchmark suite of the workflows has been released
as a community resource~\cite{dibsv1}.

%\noindent
{\bf CPS: Medium: Collaborative: CyberMech, a Novel Run-Time Substrate for 
Cyber-Mechanical Systems}
{\bf (CNS-1136073 and CNS-1136075,
WU PI C. Gill)}, 9/2011-8/2016, \$1,800,000 total.  
%
This research project developed novel foundations for parallel real-time computing, demonstrating the first ever real-time hybrid simulation involving a thousand-degree-of-freedom structure at millisecond time scales.
%
\textbf{Intellectual Merit}~-- Results of this research include new methods for parallel real-time execution of control and simulation computations, new parallel real-time scheduling techniques and analyses, and characterization and exploitation of trade-offs involving both high computational demand and stringent timing constraints.
%
\textbf{Broader Impacts}~-- This multi-university project involved 7 PhD, 3 masters, and 7 undergraduate students, and 2 visiting scholars in highly multi-disciplinary research collaborations.  Results of this research are spurring
further advances in parallel real-time computing and natural hazards
engineering.
%
\textbf{Evidence of Research Products and their Availability}~-- Results of
this 
collaborative research appeared in 10 publications at
top-tier conferences and journals.
Data, experiment configurations, platform software, and simulation source-code 
have been published on-line. % at Washington University and Purdue University.

{\bf NSF-PGRP Postdoctoral Fellowship: Improving Environmental Stress Tolerance in the Grass Models, Brachypodium distachyon and Setaria viridis}
{\bf (IOS-1202682, PI: M. Gehan)},
07/2012--06/2015, \$197,700.
%
\textbf{Intellectual Merit} -- This fellowship funded the
development of high-throughput image-based phenotyping tools and genomic
resources that aim to identify genetic loci for temperature stress and drought
response through time.
%
% in the model grass systems Brachypodium distachyon and
%Setaria viridis. 1) High-throughput image data were collected for a Setaria
%recombinant inbred line population under drought conditions as well as a
%Brachypodium accession population under combinations of drought and heat stress
%using a state-of-the-art commercial phenotyping system. 2) To analyze large
%image datasets en masse we developed open-source open-development PlantCV
%high-throughput phenotyping software through this projects to quantify plant
%traits automatically. 3) To associate phenotype data with genotype data for
%Brachypodium accessions where genotype data was not available, we generated
%genotype-by-sequencing data for 145 diploid Brachypodium accessions (Bioproject
%ID: 312869). 4) To narrow and aid in identification of stress responsive genes
%we use public circadian datasets for Brachypodium and generated high-temporal
%resolution (2H for 48H) circadian Setaria RNA-seq datasets (GSE97739; 96
%samples) to identify genes that specifically respond to day and night
%temperatures.
%
\textbf{Broader Impacts} -- This postdoctoral
fellowship directly mentored two undergraduates and one masters student, who
participated in both wet bench and computational research. PlantCV software
generated through this fellowship is also utilized as a teaching tool, and
several outreach programs were initiated including, hands-on
bioinformatics/phenomics workshops for undergraduate and graduate students,
bioinformatics workshops for K-12 teachers, and Raspberry Pi computer community
events.
%
\textbf{Availability of Data and Materials} -- Sequence data (Bioproject
ID: 312869 and GSE97739), image data, and software were distributed through
publicly accessible databases, GitHub, and Figshare. Ten publications have
resulted from this project~\cite{Bucksch17,Fahlgren+15,fgb15,Feldman18,gfa+17,ggm+15,gk17,ggg+17,hgh+17,Tovar18}.
