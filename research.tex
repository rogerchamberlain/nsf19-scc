\section{Integrative Research}
\label{sec:research}

\FIXME{Outline the technical and social science concepts and planning
activities, including potential for transferability and scalability.}

Through this proposed research, we will advance the state of the art in
catoptric systems, using two important applications to guide and evaluate the
development of prototype systems: sustainable and controlled plant growth,
and lighting
and thermal management of human-occupied buildings. Our proposed investigations
will target three primary objectives: (1) identifying and quantifying benefits
that can be achieved by redirecting light (including daylight, incandescent
light, light from LEDs, and combinations thereof) for 
plant growth (i.e., food production),
building lighting, and thermal management, through the use of
catoptric systems; (2) establishing and evaluating models and methods to ensure
the safety, reliability, maintainability, and continued efficacy of these
systems, using rigorous methods based in Markov Decision Processes for modeling
and policy generation; and (3) extracting common abstractions from the
instantiation and evaluation of the approach in three distinct application
domains (food production, building lighting, and building HVAC), as a starting
point for further application of the approach in other smart and connected
communities applications.

\subsection{Intellectual Merit}
\label{sec:im}

In addition to projected broader impacts on community nutrition, illumination
of occupied spaces, and sustainable energy use, this proposed research is of
significant intellectual merit in expanding and deepening the kinds of
cyber-physical systems capabilities that are pertinent to smart and connected
communities. Specifically, by introducing active modalities (sensing, control,
coordination, and actuation) into historically passive (daylight) or largely
static (artificial lighting) modalities, and controlling and coordinating them
with respect to well-defined objective functions in each of several distinct
applications, we seek to transform how light is used indoors.  

For example, plants growing in greenhouses often use supplemental artificial
light. Artificial lighting is energetically expensive and examining the
effectiveness of redirecting natural light in comparison to
(and possibly in combination with)
artificial lighting is both largely unexamined and well aligned with the
expertise and interests of the research team.
Similarly, daylighting (the use of natural light for illumination) design is
dominated by passive window positioning and configuration~\cite{vgf+13}
rather than
active control mechanisms, except in a few cases~\cite{kt16}.
Heating systems that
exploit sunlight frequently use actively-controlled mirrors for tracking the
relative position of the sun, but there is limited experience with using the
same sets of mirrors for both thermal and lighting control.

We propose to investigate how to utilize actively controlled catoptric (mirror)
surfaces effectively for improved food production and illumination and heating
of buildings. Through computer-based control of the dynamic positioning of
individual mirrors, and cyber-physical integration of natural and artificial
light sources, the mirrors, the devices that orient them, and customized plant
growth, building lighting and heating objectives and constraints, we propose to
enable adaptive and fine-grained management of light as a resource throughout a
building.

We propose to develop image-based maps and other interfaces to allows users of
our system to visualize zones of intensity in an interior space prior to the
mirrors redirecting light, and to allow updates to the map to increase or
decrease intensity through different modalities of redirection, filtering, or
blocking or recruiting light sources. Users also will be provided interfaces
for supplying or creating the images to be used for the map, and defining and
evaluating different objective functions for lighting intensity, thus
encouraging user-based control of, and interaction with, the system in each of
several different applications. The engagement of any member of a community in
the creation of that image in turn may impact on the entire
community~\cite{BS13} and
encourage dialog about the quality and quantity of light within the
environment.

Given the desire to control different combinations light (sunlight and
different sources of artificial light) via catoptric surfaces, for different
plant growth, illumination and/or thermal management objectives, a number of
crucial cyber-physical systems issues must be addressed.  For example, we will
investigate how existing techniques like multi-objective control be applied to
manage the relationships among potentially competing goals within each
application (which themselves also will be articulated and quantified as a
contribution of this proposed research). We also intend to investigate whether
Markov Decision Process (MDP) models can serve as a cohesive framework within
which each application domain's distinct multi-objective control problem can be
represented, and appropriate control policies generated automatically (e.g.,
through techniques like policy iteration), recognizing that maximization of an
objective function in expectation is a robust way to acknowledge the inherent
uncertainty of future events (whether it be sunlight availability, lighting
demand, or any other effect that is either stochastic in nature or is
sufficiently complex but stationary so that it can be modeled as such).

Constraints for safety, reliability, and maintainability, also will be modeled,
alongside the objective functions for the continued efficacy of a catoptric
system in each particular application domain.  Even with a suitable
multi-objective control approach in place, it is essential that such
requirements are all addressed together, at once, and consistently. For
example, considering safety: highly concentrated sunlight could be damaging
either to a plant or a person and so would lead to deterioration of the
expected value for the objective function in food production or illumination
applications.  However, in a HVAC application, such light aimed at a heat
collector (important when harvesting energy for thermal management purposes)
may need to be limited even if its increase would improve the application's
objective function, as it could be harmful to a person who inadvertently
contacted the beam of light.  This example in turn reveals the kinds of nuances
that may be in play for such systems, which in turn may inform other
cyber-physical applications - for example detecting the presence or absence of
a person in that scenario could inform sensing, control, and actuation
decisions.  Reliability and maintainability considerations similarly may span
both objective functions and constraints, for example in community food
production settings where failure of the system would impair vital nutritional
resources on which a community increasingly relies.

An important feedback mechanism we will investigate is the use of visual
data (imagery and video) for assessing the safety, reliability, and
efficacy of catoptric systems. This will include determination of the true
orientation of each mirror, providing information on the available
light and its current positioning, as well as providing feedback on
benefits to both plants and humans.
\FIXME{Abby, what, if anything, more should we say here?}

\subsection{Research Questions}
\label{sec:rq}

\FIXME{Detail technological and social science research questions, hypotheses
and research gaps that will be explored during the planning period.}

Several key considerations for this proposed research will be the effectiveness
and ease with which: (1)~abstract state representations (within the MDPs) can
adequately track and manage complex real-world applications, (2)~large numbers
of mirrors, sensors, and pan-tilt units can be managed over long periods of
time as some of them may degrade or cease functioning entirely, and
(3)~fundamental issues with observability that arise from the first two
considerations.  We propose to address the first consideration through careful
abstraction of each application domain's most salient features within our
models.  For example, in the domain of food production, we will work with our
collaborators in plant science to identify and/or define initially simplified
models for how different illumination patterns may optimize food growth - we
will then apply those models to control catoptric surfaces in small-scale
physical experiments, and refine the models based on the results of those
studies.

The second and third considerations will be addressed by tracking and reporting
the potential degradation of individual elements in the prototype test-bed we
will develop for each application domain, and examining the uncertainty in
sensing, actuation, and control that may occur because of it.  Incorporating
uncertainty into our MDP-based models, e.g., through developing POMDP models,
PAC bounds (as we have done in prior work [TBD - cite Glaubius et al. UAI
paper]), and other methods will be another important contribution of this
research.

